%------------------------
% Resume Template
% Author : Alexander Hahn
% Github : https://github.com/AlexHahnPublic (not really lol)

%------------------------

\documentclass[a4paper,20pt]{article}

\usepackage{latexsym}
\usepackage[empty]{fullpage}
\usepackage{titlesec}
\usepackage{marvosym}
\usepackage[usenames,dvipsnames]{color}
\usepackage{verbatim}
\usepackage{enumitem}
\usepackage[pdftex]{hyperref}
\usepackage{fancyhdr}

\pagestyle{fancy}
\fancyhf{} % clear all header and footer fields
\fancyfoot{}
\renewcommand{\headrulewidth}{0pt}
\renewcommand{\footrulewidth}{0pt}

% Adjust margins
\addtolength{\oddsidemargin}{-0.530in}
\addtolength{\evensidemargin}{-0.375in}
\addtolength{\textwidth}{1in}
\addtolength{\topmargin}{-.45in}
\addtolength{\textheight}{1in}

\urlstyle{rm}

\raggedbottom
\raggedright
\setlength{\tabcolsep}{0in}

% Sections formatting
\titleformat{\section}{
  \vspace{-10pt}\scshape\raggedright\large
}{}{0em}{}[\color{black}\titlerule \vspace{-6pt}]

%-------------------------
% Custom commands
\newcommand{\resumeItem}[2]{
  \item\small{
    \textbf{#1}{: #2 \vspace{-2pt}}
  }
}

\newcommand{\resumeItemWithoutTitle}[1]{
  \item\small{
    {\vspace{-2pt}}
  }
}

\newcommand{\resumeSubheading}[4]{
  \vspace{-1pt}\item
    \begin{tabular*}{0.97\textwidth}{l@{\extracolsep{\fill}}r}
      \textbf{#1} & #2 \\
      \textit{#3} & \textit{#4} \\
    \end{tabular*}\vspace{-5pt}
}


\newcommand{\resumeSubItem}[2]{\resumeItem{#1}{#2}\vspace{-3pt}}

\renewcommand{\labelitemii}{$\circ$}

\newcommand{\resumeSubHeadingListStart}{\begin{itemize}[leftmargin=*]}
\newcommand{\resumeSubHeadingListEnd}{\end{itemize}}

\newcommand{\resumeItemListStart}{\begin{itemize}}
\newcommand{\resumeItemListEnd}{\end{itemize}\vspace{-5pt}}

%-----------------------------
%%%%%%  CV STARTS HERE  %%%%%%

\begin{document}

%----------HEADING-----------------
\begin{tabular*}{\textwidth}{l@{\extracolsep{\fill}}r}
  \textbf{{\LARGE Alexander Hahn}} & Email: \href{mailto:}{afh53@cornell.edu}\\
  \href{https://github.com/AlexHahnPublic}{Github: ~~github.com/AlexHahnPublic} & Mobile:~~~(845) 248-2761 \\
\end{tabular*}

%-----------EDUCATION-----------------
\section{~~Education}
  \resumeSubHeadingListStart
    \resumeSubheading
      {Cornell University}{Ithaca, New York}
      {Bachelors - Mathematical Physics}{September 2010 - May 2014}
      %{\scriptsize \textit{ \footnotesize{\newline{}\textbf{Courses:} Operating Systems, Data Structures, Analysis Of Algorithms, Artificial Intelligence, Machine Learning, Networking, Databases}}}
    \resumeSubHeadingListEnd

\vspace{-5pt}
\section{Skills Summary}
        \resumeSubHeadingListStart
        \resumeSubItem{Languages}{~~~~~~R, Haskell, Python, Bash/Unix Commandline tools \& scripting, SQL/kdb, Scala, OCaml}
        \resumeSubItem{Technical}{~~~~~~~~Emacs, Linux, Git/Travis, Tmux, AWS (S3, EC2, Batch, CloudWatch), MOSEK, Docker, Vim, nix, IntelliJ}
        \resumeSubItem{Quantitative}{~~~Statistical Modeling, Quantitative Trading (Execution), Portfolio Construction/Analysis (constrained optimization), Data Processing, Forecasting, Scientific Computing with Linear \& Non-linear Numerical Methods}

\resumeSubHeadingListEnd
\vspace{-5pt}
\section{Experience}
\vspace{2pt}
  \resumeSubHeadingListStart
    \resumeSubheading{Bot Lab LP (Schonfeld external hedge fund)}{New York, NY}
    {Associate Quantitative Portfolio Manager/ Quant Trader}{April 2019 - Present}
    \resumeItemListStart
    \resumeItem{Overview} {Joined a Citadel quant portfolio manager and Stanford CS PhD to build a quantitive hedge fund exclusive to Schonfeld Strategic Advisors.
      Order of 100MM's initial allocation.}
    \resumeItem{Platform contributions}
               {
                 \\$\bullet$ \textbf{Trading Interface} - worked with our infrastructure engineer to implement our live trading interface.
                 Traded over 12 billion USD via 4 execution brokers over the last 3 years across $\sim$ 4000 unique publically traded US Equities.
                 As the firm's quant trader I developed models to anlayze and determine: allocation amongst brokers, allocation amongst trading algos,
                 static schedule logic (trade windows, algo instructions, auction participations), etc.
                 Had final sign off responsibility to review and submit each of Bot Lab's trading baskets to the market.
                 Additionally, reviewed execution performance with brokers (and Schonfeld) quarterly and incorporated findings into internal work/ trading decisions.
                 \\$\bullet$ \textbf{Simulation and Backtest Evaluation} - worked alongside the head PM to build a robust backtest. Specifically wrote Python, Haskell, and R
                 to analyze alpha performance over time, portfolio construction performance over time, and overall returns/ fund performance in a simulated fashion. We followed
                 a strict quantitative approach leveraging an in-sample period (2010-2015), out-of-sample period (2015-2020/Live), and hypothesis testing framework to measure the strategies
                 performance and statistically significant changes to it.
                 Main evaluation metrics were around: returns/performance, risk analysis, predictive power/ translation to return, costs (spreads, commissions, borrow rates), volumes, etc.
                 I've contributed $>$50k lines of code to the platform since inception.
               }
    \resumeItem{Research/ Strategy improvements}
               { \\$\bullet$ \textbf{Alpha} - Design and analysis of predictions across 5 thematic buckets: Market Data, Fundamental, Sentiment, Microstructure, and
                 Cross Asset. I determined what frequencies our alphas were best at predicting and incorporated that info into pc/trading to boost performance.
                 This resulted in a three window daily trading model greatly increasing our monetization rate at the trade off of higher trading volumes.
                 Similarly, I implemented and automated many studies to identify asymmetries to incorperate back into the pipeline.
                 \\$\bullet$ \textbf{CBOE Options Example} - Performed initial analysis of the CBOE trial data set in R. Upon determining that data set was worth
                 incorporating into our signals I wrote the file ingestion, parsing, and cleaning code in python, then calculated contract level statistics including, greeks,
                 implied vol, implied price, open interest, volumes, etc. From there I wrote the aggregation functions into symbol level statistics (features) that our alpha would use and fit on.
                 Regressing the Options signal on the master signal (vs other signals) this sub alpha accounts for roughly 20$\%$ of our overall predictive power. This is a
                 decently outsized contribution given there are $\sim$10 signals overall.
                 \\$\bullet$ \textbf{Risk Modeling and Portfolio Construction}: Worked with the head PM to construct and implement the portfolio mathematics
                 to pass our predictions into a conic optimization program. Reduced risk by modeling new market phenomena (designing factors on Elections, Covid, China, Russia/Ukraine) and passing them
                 into the optimization software as a constraint to bound exposure to. Additionally, I ran many portfolio simulations to analyse and fine tune our portfolio construction process,
                 especially in cases where anything upstream (alpha, data, etc) changed.
               }

    \resumeItemListEnd
    \vspace{2pt}
    \resumeSubheading
        {Bridgewater Associates}{Glendinning/Westport, CT}
        {Investment Engineer \& Production Integration Engineer}{Dec 2016 -  March 2019}
    \resumeItemListStart
    \resumeItem{Macro Economic Model Updates}
               {Systemized Volume, Open Interest, and Greek Estimates of Brazilian options/futures contracts into Bridgewater's Data and Backtesting platform.
                 Previously the system was a mixture of Excel and manual bloomberg data entry.
                 I overhauled the system to improve and port the logic into Bridgewater's Scala based production platform.}
%    \resumeItem{Investment Engineering Course}
%               {All Investment engineers are given a course which overviewed Asset classes, Macro Economics, Portfolio Theory,
%                 and Bridgewater's framework for evaluating non consenus prices/returns.}
     \resumeItem{Production Integration Engineer - Team Lead}
                {Led a team of six Engineers to assist
                  the Research Department in utilizing Bridgewater's main Backtesting/Signal
                  Generation Platform (Lightspeed), and diagnostic tools (Fusion - advanced UI
                  charting tool). Supported Production Modules and operations team in running
                  live/ production issues. Supported Researchers' various use cases and implementations of logic into the platform.
                  Added functionality to the Investement Engineer/Associate userfacing API as needed.}
    \resumeItemListEnd
    \vspace{2pt}
    \resumeSubheading
        {Nomura}{New York, NY}
        {Technology Analyst}{August 2014 -  November 2016}
        \resumeItemListStart
        \resumeItem{Automation and Support of Structured Products}
                   {Worked on a team to systemize Nomura's Equity Linked Bond business enabling the growth of the book.
                     Helped implement the STP (straight through processing) of notes from
                     issuance to expiration (or redemption). This included pricing/valuation, periodic coupon/ payment analysis, risk analysis / hedging (index and dynamic).
                     Worked alongside a trader and desk quant to help support the growth of the  portfolio to 24 large notional Notes with tenors varying from 1-10 years.}
        \resumeItem{Security Master}{Supported Nomura's Reference database. Rationalized and help maintain consistency between numerous security identifiers. }
        \resumeItemListEnd

        \resumeSubHeadingListEnd

\end{document}
