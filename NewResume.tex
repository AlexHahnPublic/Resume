% LaTeX file for resume
% This file uses the resume document class (res.cls)

\documentclass{res}
%\usepackage{helvetica} % uses helvetica postscript font (download helvetica.sty)
%\usepackage{newcent}   % uses new century schoolbook postscript font
\newsectionwidth{0pt}  % So the text is not indented under section headings
\setlength{\textheight}{10.2in} % set text height big enough for box
\topmargin=-.8in       % to start box .5in from top of page
\oddsidemargin=-.5in   % to start box .5in from left of page
\usepackage{multicol}
\begin{document}

%%%%%%%%%%%%%%%%%%%%%%%%%%%%%%%%%%%%%%%%%%%%%%%%%%%%%%%%%%%%%%%%%%%%%%%%%%%%
% The following lines define \boxaround, used to draw a box on the page.
% The parameter is the entire text of the resume. Must fit on one page!
%
% \boxaroundhmargin is the left & right margin around the text inside the box.
% \boxaroundvmargin is the top & bottom margin around the text inside the box.
% \boxrulethickness controls thickness of line used to draw the box.
% You can change these 3 things in the lines below:
%%%%%%%%%%%%%%%%%%%%%%%%%%%%%%%%%%%%%%%%%%%%%%%%%%%%%%%%%%%%%%%%%%%%%%%%%%%%%
\newdimen\boxrulethickness\newdimen\boxaroundhmargin\newdimen\boxaroundvmargin
\boxrulethickness=.5pt        %controls thickness of line
\boxaroundhmargin=35pt        % about a half inch
\boxaroundvmargin=40pt        % to fit more text on page, make this smaller
%%%%%%%%%%%%%%%%%%%%%%%%% Don't read this stuff %%%%%%%%%%%%%%%%%%%%%%%%%%%%%%
\hsize=7.5in% \vsize=10.5in             % use bigger dimensions for box
\newbox\MACboxA  \newdimen\MACdimenA
% \borderandboxit is used inside \boxaround:
\def\borderandboxit#1#2#3{\vbox{\hrule height#2\hbox{\vrule width#2\hskip#1\hskip-#2%
  \vbox{\vskip#1\relax#3\vskip#1}\hskip#1\hskip-#2\vrule width#2}\hrule height#2}}
%
\long\def\boxaround#1{\vskip6pt
  {\MACdimenA=\hsize \advance\MACdimenA by-\boxaroundhmargin
   \advance\MACdimenA by-\boxaroundhmargin   % once for each side
   \setbox\MACboxA=\hbox to \hsize{\hskip\boxaroundhmargin%\hss
                     \vbox{\hsize=\MACdimenA
                           \vskip\boxaroundvmargin #1
                           \vskip\boxaroundvmargin}\hss}%
   \borderandboxit{0pt}\boxrulethickness{\box\MACboxA}}%
  \vskip2pt plus0pt minus0pt
}
%%%%%%%%%%%%%%%%%%%  End of \boxaround macro %%%%%%%%%%%%%%%%%%%%%%%%%%%%%%%%%

\boxaround{ % put the text on the page inside a box

\name{Alexander Ford Hahn\\[12pt]}
\address{\bf Present Address\\505 West 54th Street Apt 1116 Joy Drive, New
York, NY, 10018\\
email- Afh53@physics.cornell.edu, phone (845) 248-2761}
\address{\bf Permanent Address\\117 Joy Drive\\
         Valley Cottage, NY 10989}


\begin{resume}

\section{\sl\bf Experience\\}


\begin{ncolumn}{2}

{\it Nomura Securities International, Technical Analyst} & ~~~~~~~~~~~~~~~~~~~~~~~~~~~~~~~~~~~~~~~~~~~~~~~~~~~~~~~~8/13-Present
\end{ncolumn}\\
New York, NY\\
Currently working on the second leg of a 1.5 year long automation project.
First phase focused on Nomura America's Fixed Income trading desk for OTC
derivatives. Current phase is centered around Equity linked OTC
derivatives. Another larger task of mine is reworking Nomura America Finance
LLC's FI and Equity linked notes trade processing and management structure.
Both projects are critical to achieve an overarching legacy mainframe
decommission project and realize the Straight Through Automated Processing
golden standard enabling much higher trade volume and efficiency for numerous products and valuation types.

\begin{ncolumn}{2}
{\it Cornell Mathematics Department, teaching assistant} & ~~~~~~~~~~~~~~~~~~~~~~~~~~~~~~~~~~~~~~~~~~~~~~~~~~~~~~~~8/13-Present
\end{ncolumn}\\
Cornell, Ithaca NY\\
Two years teaching experience with over 100 undergraduates. Main courses included, Calculus 1 \& 2, Linear Algebra, Multivariable Calculus, and intro MATLAB. As per my concentration in mathematical physics I was allocated more towards the physics undergrads and their relevant courses: ODE and PDE solver techniques, linear algebra, Lie Theory, and Matrix computation algorithms (Numerical Methods).

\begin{ncolumn}{2}
{\it Physics Teacher, and OCSEP Ambassador}  &   ~~~~~~~~~~~~~~~~~~~~~~~~~~~~~~~~~~~~~~~~~~~~~~~~~~~~~~~~5/13-8/13
\end{ncolumn}\\
Oxbridge China Student Education Program (OCSEP), Shanghai China\\
TA and Grading responsibilities for an Intro to Mechanics Course geared towards Chinese International Students in Shanghai.


\section{\sl\bf  Education}
Cornell University B.S. May 2014,
\textbf{Majors}: Mathematics, Physics \textbf{Concentrations}: Mathematical
Physics, Quantum Physics, Scientific Computing
Nyack High School (2006-2010)
National Merit Scholar

\section{\textbf{Coursework}}
({\bf Languages/technologies proficient by skill level}:
MATLAB, Python, LaTeX, SQL, bash/ shell scripting, Vim, OCaml Mathematica, (some C++ but mainly in the context of scientific computing/Numerical Analysis, implementation of many of ``Numerical Recipes" by professor Teukolsky and Van Loan)

\begin{multicols}{3}
\begin{itemize}
\item{Quantum Physics}
\item{Functional programming\\
and Data\\
Structures (OCaml)}
\item{Quantum Information\\
Processing*\\
(quantum computing)}
\item{Matrix Lie Groups}
\item{Data Structures \&\\
Object Oriented Programming (Java)}
\item{Numerical Analysis (Linear and Nonlinear EQ's)}
\item{Computational\\
Physics* (C++, Mathematica)}
\item{Honors Intro Analysis}
\item{Numerical Analysis (ODE's and PDE's)}
\item{Multivariable Calculus for Engineers}
\item{Linear Algebra}
\item{Special Relativity}
\item{MATLAB}
\item{Exoplanetary Systems}
\item{Electricity and Magnetism}
\item{Mechanics}
\item{Two semesters of French}
\item{Thermodynamic and Statistical Physics}
\item{Matrix Computations* (CS)}
\item{Applicable Algebra}

\end{itemize}
\end{multicols}
\begin{center}
(* \textit{denotes graduate (PhD) level courses})
\end{center}
\section{\sl\bf  Honors \& Activities}
\begin{ncolumn}{1}
National Merit Scholar, Cornell Mathematical Modeling Competition (MCM), Cornell Symphony Orchestra, Cornell United Club Soccer, Cornell Math and Physics Club, Association of CS Undergraduates
\end{ncolumn}
\end{resume}
\vfill} %   end the material being boxed.
\end{document}



