% ###################################################################
% ##################### Alex Hahn's Resume ##########################
% ###################################################################


% This file uses the resume document class (res.cls)

\documentclass{res}
%\usepackage{helvetica} % uses helvetica postscript font (download helvetica.sty)
%\usepackage{newcent}   % uses new century schoolbook postscript font
\newsectionwidth{0pt}  % So the text is not indented under section headings
\setlength{\textheight}{10.2in} % set text height big enough for box
\topmargin=-.76in       % to start box .5in from top of page
\oddsidemargin=-.77in   % to start box .5in from left of page
\usepackage{multicol}
\usepackage{xcolor}
\usepackage[colorlinks = true,
            linkcolor = blue,
            urlcolor  = blue,
            citecolor = blue,
            anchorcolor = blue]{hyperref}

\newcommand{\MYhref}[3][blue]{\href{#2}{\color{#1}{#3}}}%


%\setlength\columnsep{90pt} % For the spacing of cols in the coursework section
\usepackage{hyperref}
\usepackage{changepage}
\begin{document}

%%%%%%%%%%%%%%%%%%%%%%%%%%%%%%%%%%%%%%%%%%%%%%%%%%%%%%%%%%%%%%%%%%%%%%%%%%%%
% The following lines define \boxaround, used to draw a box on the page.
% The parameter is the entire text of the resume. Must fit on one page!
%
% \boxaroundhmargin is the left & right margin around the text inside the box.
% \boxaroundvmargin is the top & bottom margin around the text inside the box.
% \boxrulethickness controls thickness of line used to draw the box.
% You can change these 3 things in the lines below:
%%%%%%%%%%%%%%%%%%%%%%%%%%%%%%%%%%%%%%%%%%%%%%%%%%%%%%%%%%%%%%%%%%%%%%%%%%%%%
\newdimen\boxrulethickness\newdimen\boxaroundhmargin\newdimen\boxaroundvmargin
\boxrulethickness=.5pt        % controls thickness of line
\boxaroundhmargin=25pt        % changes distance from box to text
                              % horizontally
\boxaroundvmargin=11pt         % changes distance from box to text
                              % vertically (from top)
%\hsize=7.5in% \vsize=10.5in             % use bigger dimensions for box
\hsize=8in% \vsize=10.5in             % use bigger dimensions for box
\newbox\MACboxA  \newdimen\MACdimenA
% \borderandboxit is used inside \boxaround:
\def\borderandboxit#1#2#3{\vbox{\hrule height#2\hbox{\vrule width#2\hskip#1\hskip-#2%
  \vbox{\vskip#1\relax#3\vskip#1}\hskip#1\hskip-#2\vrule width#2}\hrule height#2}}
%
\long\def\boxaround#1{\vskip6pt
  {\MACdimenA=\hsize \advance\MACdimenA by-\boxaroundhmargin
   \advance\MACdimenA by-\boxaroundhmargin   % once for each side
   \setbox\MACboxA=\hbox to \hsize{\hskip\boxaroundhmargin%\hss
                     \vbox{\hsize=\MACdimenA
                           \vskip\boxaroundvmargin #1
                           \vskip\boxaroundvmargin}\hss}%
   \borderandboxit{0pt}\boxrulethickness{\box\MACboxA}}%
  \vskip2pt plus0pt minus0pt
}
%%%%%%%%%%%%%%%%%%%  End of \boxaround macro %%%%%%%%%%%%%%%%%%%%%%%%%%%%%%%%%

\boxaround{ % put the text on the page inside a box

\name{~~~~~~~Alexander Ford Hahn\\[12pt]}
\address{\bf Contact Info and Profiles\\Phone - (845) 248-2761\\
Email - Afh53@Cornell.edu\\
{\bf GitHub}: \MYhref{https://github.com/AlexHahnPublic?tab=repositories} {https://github.com/AlexHahnPublic?tab=repositories}}
\address{\bf ~~~~~~~~~~~~~~~~~~~~~Permanent
Address\\~~~~~~~~~~~~~~~~~~~~~~~~117 Joy Drive\\
         ~~~~~~~~~~~~~~~~~~~~~~~~Valley Cottage, NY 10989}

\begin{resume}

\section{\sl\bf Experience\\}


\begin{ncolumn}{2}

{\it Nomura Securities International, Tech Analyst} & ~~~~~~~~~~~~~~~~~~~~~~~~~~~~~~~~~~~~~~~~~~~~August '14 - August '16
\end{ncolumn}\\
New York, NY\\
Recently finished the third leg of a 1.5 year long automation project.
Phase I automated the Nomura America's Fixed Income OTC Derivatives trading
desk front to back flow. From pricing/ valuation functions, Cash and
Settlement processes, to record keeping and financial ledger postings, I was a tech
analyst working alongside numerous areas of the firm. Phase II focused on Nomura America's Equity OTC
Derivatives trade flow. With the completion of the first phase and team
re-orgs I was promoted to be the lead tech analyst of the project and managed the entire project through the SDLC project framework. Similar to phase I I worked with the full spectrum
of the firm, from the derivative quants, traders, Risk, Ops, Product
Controllers, to a multitude of tech teams. Some more common tasks included
investigating and updating pricing and valuation models, constructing xml messages and file feeds, creating
test scripts/ harnesses, Oracle SQL table design, and other FOBO automations
for full STP. The completion of NHA Derivatives Phase I\&II
automation helped Nomura increase operational efficiency and reduce/ refactor
headcount. Phase III focused more on automating complex exotic notes and
bonds, hedged with various derivative swap models to create custom tailored
financial products for clients.\\

\begin{ncolumn}{2}
{\it Cornell Mathematics Department, Teaching Assistant} & ~~~~~~~~~~~~~~~~~~~~~~~~~~~~~~~~~~~~~~~~~~~~~~~~August '13 - May '14
\end{ncolumn}\\
Cornell, Ithaca NY\\
Two years math center teaching experience with over 100 undergraduates. The majority came for Calculus 1 \& 2, Linear Algebra, Multivariable Calculus, and Differential Equations. As per my concentration in mathematical physics/scientific computing I was allocated more towards the physics undergrads and their relevant courses: ODE and PDE solver techniques, linear algebra, Lie Theory, and Matrix computation algorithms (Numerical Methods). Furthermore, I TA'ed a mechanics 101 course and assisted in laser cavity cooling research during the summer of 2013 through Cornell at The University of Shanghai for Science and Technology.

\section{\sl\bf  Education}
Cornell University B.A. May 2014,
\textbf{Major}: Mathematics \textbf{Concentrations}: Mathematical
Physics, Scientific Computing, Numerical Analysis, \textbf{High School}: Nyack High School (2006-2010) National Merit Scholar

\section{\textbf{Skills/Technology}}
{\bf General Purpose Languages (by experience)}: Python, C++, OCaml\\
{\bf Domain Specific Languages}: MATLAB, LaTeX, SQL, Bash/Shell Mathematica\\
{\bf Other}: Advanced VIM user, Emacs for OCaml for various reasons, advanced
excel user\\
Implementation of many of ``Numerical Recipes in C++" by Professor Teukolsky and "Matrix Computations" by Professor Van Loan)

\section{\textbf{Coursework \textmd{~~~~~~~~~~~~~~~~~~~~~~~~~~~~~~~~~~~~(* \textit{denotes graduate (PhD) level courses})}}}
\begin{adjustwidth}{2.4em}{-65pt}
\begin{multicols}{3}
\begin{itemize}
\item{Quantum Physics (various)}
\item{Functional programming and Data Structures (OCaml)}
\item{Quantum Information\\
Processing*\\
(quantum computing)}
\item{Matrix Lie Groups}
\item{Data Structures \&
Object Oriented Programming (Java)}
\item{Numerical Analysis (Linear and Nonlinear EQ's, MATLAB)}
\item{Computational\\
Physics* (C++, Mathematica)}
\item{Honors Intro to Mathematical Analysis}
\item{Numerical Analysis (ODE's and PDE's, MATLAB)}
\item{Multivariable Calculus for Engineers}
\item{Linear Algebra}
\item{General \& Special Relativity}
\item{MATLAB}
\item{Techniques in Exoplanetary Systems Detection}
\item{Electricity and Magnetism}
\item{Mechanics \& Kinematics}
\item{Thermodynamic and Statistical Physics}
\item{Matrix Computations* (CS, MATLAB)}
\item{Number Theory}
\end{itemize}
\end{multicols}
\end{adjustwidth}
\section{\sl\bf  Honors \& Activities}
\begin{ncolumn}{1}
National Merit Scholar, Cornell Mathematical Modeling Competition (MCM), Cornell Symphony Orchestra, Cornell United Club Soccer, Cornell Math and Physics Club, Association of CS Undergraduates
\end{ncolumn}
\end{resume}
\vfill} %   end the material being boxed.
\end{document}


